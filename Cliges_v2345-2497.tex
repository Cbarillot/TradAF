\documentclass[12pt,a4paper]{article}
\usepackage[utf8]{inputenc}
\usepackage[T1]{fontenc}
\usepackage[french]{babel}
\usepackage{tradaf}
\usepackage{hyperref}

\title{Traduction Juxtalinéaire et Analyse Philologique\\
\large Chrétien de Troyes, \textit{Cligès}\\
v. 2345--2497}
\author{D'après Valérie Naudet}
\date{\today}

\begin{document}

\maketitle

\section*{Référence}

Édition bilingue de Laurence Harf-Lancner, Paris, Champion, Champion Classiques Moyen Âge, 2006.

\section*{Abréviations}

Quelques abréviations, outre celles courantes et non relevées dont on use habituellement en grammaire ou lexicologie : CS/CR (cas sujet/cas régime), F/M (féminin/masculin), S/P (singulier/pluriel), le tout pouvant être combiné (CSFP : cas sujet féminin pluriel), CRA cas régime absolu ; PS passé simple, IP, présent de l'indicatif, suivis d'un chiffre renvoyant à la personne grammaticale ; PPS / PPC pronom personnel sujet ou complément.

Les mentions suivantes peuvent également être trouvées : \textit{gl} indique que la glose a été trouvée dans le glossaire de notre édition au programme, \textit{DECT} dans le dictionnaire électronique de Chrétien de Troyes (\url{http://zeus.atilf.fr/dect/}).

\section*{Éditions consultées}

\begin{itemize}
  \item Chrétien de Troyes, \textit{Cligés}, éd. et trad. Philippe Walter, dans Chrétien de Troyes, \textit{Œuvres complètes}, Daniel Poirion dir., Paris, Gallimard, « Bibliothèque de la Pléiade », 1994.
  \item Chrétien de Troyes, \textit{Cligés}, éd. et trad. Charles Méla et Olivier Collet, dans Chrétien de Troyes, \textit{Romans}, Paris, Le Livre de Poche, « La Pochothèque », 1994.
\end{itemize}

\newpage

\begin{translationsection}{v. 2345--2350}

\begin{translation}

\oldfrench
\verseline Cil li a le conte randu.\notelexique{un jor : renforcer dans la traduction la valeur cardinale de l'article indéfini, proche ici de son étymon latin, le cardinal unus, unam, unum.}\\
\verseline Et li rois n’a plus atandu\notelexique{enor, s.m. : « marque de respect et d'estime, faveur » (DECT).}\\
\verseline que lors n’an face sa justise.\\
\verseline Mes molt loe Alixandre et prise,\\
\verseline et tuit li autre le conjoent\notesyntaxe{del chastel et de ce que : donner à la préposition de son sens latin « au sujet de ».}\\
\verseline et formant le prisent et loent.\notesyntaxe{li rois Artus est le sujet de promist dont le COD est la complétive (mise dans l'ordre des mots usuel aujourd'hui) qu'il li donroit le meillor reiaume de Gales quant sa guerre avroit finee.}\\
\noteother{L'auteur privilégie ici la justice royale immédiate face à la trahison d'Angrés.}\notemorphologie{Le verbe 'conjoent' (conjoïr) montre la persistance du préfixe d'accompagnement. Emploi du cas sujet pluriel 'li autre'.}
\modernfrench
et lui livre le comte. Le roi en fait justice sans plus attendre. Il couvre Alexandre d’éloges et tous le fêtent et le félicitent, tous pleins de joie, sans exception.

\end{translation}

\end{translationsection}

\begin{translationsection}{v. 2351--2355}

\begin{translation}

\oldfrench
\verseline N'i a nul qui joie ne maint;\\
\verseline por la joie li diax remaint\\
\verseline que il demenoient einçois,\\
\verseline mes a la joie des Grezois\notelexique{le jor : l'article défini a ici une valeur forte, démonstrative (Claude Buridant, Grammaire nouvelle de l'ancien français, Paris, Sedes, 2000, p. 120), qu'il faut impérativement rendre dans la traduction. Il s'agit d'un souvenir de son étymon lati...}\\
\verseline ne se pot autre joie prandre.\\
\notesyntaxe{Alternance de 'joie' et 'diax' (deuil) marquant le contraste sémantique fréquent dans le roman.}
\modernfrench
La joie succède au deuil qui régnait auparavant. Mais nulle joie ne peut se comparer à celle des Grecs.

\end{translation}

\end{translationsection}

\begin{translationsection}{v. 2356--2362}

\begin{translation}

\oldfrench
\verseline Li rois li fet la cope randre\\
\verseline de .XV. mars, qui molt fu riche,\notemorphologie{graindre : comparatif de supériorité CSFS de grand. de ce que : voir supra à propos de la préposition de. Fierce, s.f. : « reine du jeu d'échec » Fu fierce et fu rois : noter la valeur descriptive du PS en ancien français, impossible à garder en l...}\\
\verseline et si li dit bien et afiche\\
\verseline qu’il n’a nule chose tant chiere,\notesyntaxe{Selon le DECT, la grainne renvoie à ce qui, chez la femme, correspond au sperme masculin. C'est la rencontre des deux qui permet la germination d'où le fruit, l'enfant, naîtra. Toutefois, les trois traductions consultées, y compris celle de Lauren...}\\
\verseline se il fet tant qu’il la requiere,\\
\verseline fors la corone et la reïne,\\
\verseline que il ne l’an face seisine.\\
\noteother{Le 'don contraignant' est une convention littéraire arthurienne où le roi s'engage sans connaître la requête (cf. note 8).}\notesyntaxe{Le terme 'seisine' est un terme juridique féodal désignant la prise de possession.}
\modernfrench
Le roi fait remettre à Alexandre la précieuse coupe de quinze marcs; il affirme et répète au jeune homme qu’il peut lui demander son bien le plus cher (hormis la couronne et la reine): ce bien lui sera livré.

\end{translation}

\end{translationsection}

\begin{translationsection}{v. 2363--2370}

\begin{translation}

\oldfrench
\verseline Alixandres de ceste chose\\
\verseline son desirrier dire n’en ose,\\
\verseline et bien set qu’il n’i faudroit mie\\
\verseline se il li requeroit s’amie.\\
\verseline Mes tant crient qu’il ne despleüst\notemorphologie{Cligés en cui memoire : CRA construit avec le pronom relatif cui ayant Cligés pour antécédent placé, comme attendu dans ce cas, devant le nom qu'il détermine.}\\
\verseline celi qui grant joie en eüst,\notelexique{Prendre roman, s.m., dans son sens premier de langue vulgaire par opposition au latin, langue savante (gl).}\\
\verseline que molt mialz se vialt il doloir\notelexique{Le vasselage, s.m., désigne l'ensemble des qualités qui font d'un homme un bon vassal pour son seigneur, soit essentiellement la prouesse aux armes, la chevalerie, le courage, la bravoure.}\\
\verseline que il l’eüst sor son voloir.\\
\noteother{Illustration de la 'fin'amor' où le désir de la dame prime sur celui de l'amant.}\notesyntaxe{Utilisation de 'vialt' (vouloir) avec diphtongaison du 'e' entravé typique de la copie de Guiot (cf. fiche graphie V.5).}
\modernfrench
Pourtant Alexandre n’ose dire son désir, bien qu’il sache qu'il serait exaucé, s’il demandait son amie. Mais il craint tant de déplaire à celle qui, au contraire, en aurait eu une grande joie, qu’il aime mieux souffrir que de l’obtenir malgré elle.

\end{translation}

\end{translationsection}

\begin{translationsection}{v. 2371--2379}

\begin{translation}

\oldfrench
\verseline Por ce respit quiert et demande\\
\verseline qu’il ne vialt feire sa demande\noteother{dire et conter : le roman présente un doublet de termes quasi synonymes qu'il convient impérativement de traduire dans le cadre d'une version d'agrégation en mettant en évidence la relation qui les unit, ici celle d'un hyperonyme dire par rapport ...}\\
\verseline tant qu’il an sache son pleisir.\\
\verseline Mes a la cope d’or seisir\notelexique{venir a sa fin, expr. : « mourir ».}\\
\verseline n’a respit n’atendue prise:\\
\verseline la cope prant et par francise\\
\verseline proia monsignor Gauvain\notemorphologie{pot : PS3 pooir.}\\
\verseline tant qu’il de lui cele cope prant,\notemorphologie{einz, préposition de temps : « avant » amasser, v.tr. : peut avoir en ancien français un régime animé humain, il prend lors le sens de « rassembler, réunir » baron (B1 ber, B2 baron, s.m. de la 3^e déclinaison) : faux ami. Si baron aujourd'hui est...}\\
\verseline mais a molt grant paine l’a prise.\\
\notemorphologie{Les vers entre crochets sont absents du manuscrit A (Guiot) et corrigés d'après les autres sources (cf. note 19).}\notesyntaxe{Élision du 'e' dans 'qu'il' devant voyelle. La graphie 'francise' (franchise) pour la noblesse de caractère.}
\modernfrench
Il demande donc un délai car il ne veut pas faire sa demande sans connaître les sentiments de la jeune fille. Mais il prend sans attendre la coupe d’or et dans un geste plein de noblesse, il supplie monseigneur Gauvain d’accepter de lui la coupe, que celui-ci finit par accepter, non sans peine.

\end{translation}

\end{translationsection}

\begin{translationsection}{v. 2380--2387}

\begin{translation}

\oldfrench
\verseline Qant Soredamors a aprise\\
\verseline d’Alixandre voire novele,\notesyntaxe{Les deux relatives introduites par ou ont toutes deux pour antécédent Bretaigne et c'est à Alexandre que renvoie le sujet de leurs verbes. La construction juxtaposée est aujourd'hui peu heureuse => rétablir un lien de coordination entre les deux s...}\\
\verseline molt li plot et molt li fu bele.\\
\verseline Qant cele sot que il est vis,\notemorphologie{murent : PS6 movoir acuellent : IP6 accuellir au sens de « commencer ».}\\
\verseline tel joie en a qu’il li est vis\\
\verseline que ja mes n’ait pesance une ore;\notemorphologie{tormante, s.f. : « tempête » à la rime avec tormante IP3 de tormanter, v.tr., « accabler, malmener ».}\\
\verseline mes trop ce li sanble demore\\
\verseline que il ne vient si com il sialt.\notemorphologie{tuit : pron. indéfini, CSPM de tout.}\\
\notemorphologie{Le verbe 'sialt' (soloir) est une forme archaïque pour 'a l'habitude'.}
\modernfrench
Quant à Soredamor, elle est remplie de bonheur en apprenant la vérité sur Alexandre. En apprenant qu’il est vivant, elle ressent une telle joie qu’elle en oublie tout son chagrin ; mais elle languit de voir qu’il ne vient pas la voir comme à son habitude.

\end{translation}

\end{translationsection}

\begin{translationsection}{v. 2388--2393}

\begin{translation}

\oldfrench
\verseline Par tans avra ce qu’ele vialt\notemorphologie{fors, prep. : expression de l'exception. Felon, s.m., CSS : entre en relation de quasi synonymie avec renoié pour désigner un homme qui trahit sa parole et son seigneur.}\\
\verseline car anbedui par contançon\notemorphologie{menor et graignor : deux comparatifs de supériorité au CRS, respectivement de petit et de grand.}\\
\verseline sont d’une chose an grant tançon.\\
\verseline Molt estoit Alixandre tart\\
\verseline que seulemant d’un dolz regart\noteother{s'an est retornez et dit ont le parjure pour sujet.}\\
\verseline se poïst a leisir repestre.\\
\notemorphologie{Subjonctif 'poïst' exprimant le souhait non encore réalisé.}
\modernfrench
Bientôt elle obtiendra satisfaction car tous deux rivalisent d’impatience pour la même chose. Alexandre avait hâte de pouvoir se repaître ne serait-ce que d’un doux regard.

\end{translation}

\end{translationsection}

\begin{translationsection}{v. 2394--2399}

\begin{translation}

\oldfrench
\verseline Grant piece a que il volsist estre\\
\verseline el tref la reïne venuz,\\
\verseline se aillors ne fust detenuz.\notelexique{lor seigneur : désigne ici Alexandre, soi disant mort avec le reste de l'équipage et des passagers dans le naufrage dû à la tempête.}\\
\verseline La demore molt li desplot;\notemorphologie{ne... mes que : expression de l'exception ; il renvoie au renégat, la forme prédicative et tonique du PPS étant explicable par le caractère elliptique de la proposition exceptive.}\\
\verseline au plus tost que il onques pot\\
\verseline vint a la reïne an son tré.\notemorphologie{cil, sujet de fu creüz, a pour référent le renégat ; noter le féminin de mensonge en langue médiévale.}\\
\noteother{La correction de 'tref' en 'tré' (v. 2244) suit la rime avec 'encontré' du vers suivant.}\notemorphologie{Passé simple 'volsist' utilisé comme conditionnel.}
\modernfrench
Depuis longtemps il serait venu dans la tente de la reine, s’il n’avait été retenu ailleurs : ce retard le fait enrager. Dès qu’il le peut, il vient rejoindre la reine dans sa tente.

\end{translation}

\end{translationsection}

\begin{translationsection}{v. 2400--2404}

\begin{translation}

\oldfrench
\verseline A Guinesores en un jor\notelexique{chalonge, s.m. : « contestation » (gl). C'est de la même racine morphologique que descend le challenge que l'anglais nous a renvoyé.}\\
\verseline ot Alixandres tant d’enor\notesyntaxe{si : adverbe de phrase dont la principale fonction est la saturation de la place 1 de la proposition et le lien avec ce qui précède. Descendant de sic latin, il n'a aucune valeur hypothétique (i.e. traduction par si hypothétique => contresens grav...}\\
\verseline et tant de joie con lui plot.\\
\verseline Trois joies et trois enors ot:\noteother{ne tarda mie : le sujet impersonnel n'est pas exprimé => à rétablir.}\\
\verseline l’une fu del chastel qu’il prist,\\
\notesyntaxe{v. 2345: 'un jor', renforcer la valeur cardinale de l'article indéfini (unus). v. 2346: 'enor', marque de respect et faveur. v. 2349: 'del chastel', donner à 'de' son sens latin 'au sujet de'.}\notemorphologie{Guinesores : Nom propre difficile à classer. Un jor : Article indéfini masculin singulier CR. Ot : Passé simple P3 du verbe avoir (forme B5+t).}
\modernfrench
À Windsor, en une seule journée, Alexandre connut autant d'honneur et de joie qu'il lui plut. Il eut trois joies et trois honneurs : l'un fut le château qu'il prit,

\end{translation}

\end{translationsection}

\begin{translationsection}{v. 2405--2408}

\begin{translation}

\oldfrench
\verseline l’autre de ce que li promist\notemorphologie{sot : PS3 savoir.}\\
\verseline li rois Artus qu’il li donroit,\\
\verseline quant sa guerre finee avroit,\notemorphologie{Rétablir un ordre des mots fluide et conforme à la phrase moderne, éviter à tout prix le calque, syntaxique et/ou lexical dans des vers qui sont relativement transparents quant à leur sens. Le futur voldra (P3 voloir) peut poser problème dans la c...}\\
\verseline le meillor reiaume de Gales ;\\
\notesyntaxe{v. 2350-53: Construction de la complétive dépendante de 'promist'. 'De ce que': voir préposition 'de' au sens de 'au sujet de'.}\notemorphologie{Promist : Passé simple P3 fort sigmatique (B5+t). Le meillor reiaume de Gales : Construction prépositionnelle du complément déterminatif (de + non animé).}
\modernfrench
l'autre de ce que le roi Arthur lui promit qu'il lui donnerait, quand sa guerre serait finie, le meilleur royaume de Galles ;

\end{translation}

\end{translationsection}

\begin{translationsection}{v. 2409--2412}

\begin{translation}

\oldfrench
\verseline le jor l’en fist roi an ses sales.\notelexique{destorber, v.tr. : « détourner, empêcher » (gl).}\\
\verseline La graindre joie fu la tierce,\notemorphologie{einçois, adv. : marquant une rectification et un renchérissement : il ouvre un énoncé qui, renchérissant sur le précédent, ne s'oppose à lui que sur la formulation : dire que le roi n'entrave en rien le projet d'Alexandre va dans le même sens que ...}\\
\verseline de ce que s’amie fu fierce\\
\verseline de l’eschaquier dom il fu rois.\\
\notelexique{v. 2354: 'le jor', l'article a une valeur démonstrative forte. 'Sale' est un faux ami (pièce principale ou palais). v. 2357: 'graindre' est un comparatif de supériorité. 'Fierce': reine du jeu d'échec.}\notesyntaxe{Graindre : Adjectif à alternance de bases (B1), comparatif synthétique issu de 'grandior'. Fierce de l'eschaquier : Complément du nom avec 'de'.}
\modernfrench
ce jour-là, il le fit roi en ses salles. La plus grande joie fut la troisième, du fait que son amie fut la reine de l'échiquier dont il fut le roi.

\end{translation}

\end{translationsection}

\begin{translationsection}{v. 2413--2416}

\begin{translation}

\oldfrench
\verseline Einz que fussent passé troi mois,\\
\verseline Soredamors se trova plainne\\
\verseline de semance d’ome et de grainne,\notemorphologie{se lui pleüst //, grant force menee en eüst : système hypothétique au subjonctif imparfait (pleüst et eüst, respectivement Sub.impft P3 de plaire et avoir) à valeur de potentiel.}\\
\verseline si la porta jusqu’à son terme.\\
\notesyntaxe{v. 2359-64: 'grainne' et 'semance' sont construits comme compléments de l'adjectif 'plainne'. 'Estre en son germe' signifie germer.}\notemorphologie{Einz que : Locution conjonctive d'antériorité régissant le subjonctif. Semance d'ome : Construction avec 'de' car le complément est un nom de personne indéterminé.}
\modernfrench
Avant que trois mois fussent passés, Soredamors se trouva enceinte de la semence d'un homme et de son germe, et elle le porta jusqu'à son terme.

\end{translation}

\end{translationsection}

\begin{translationsection}{v. 2417--2420}

\begin{translation}

\oldfrench
\verseline Tant fu la semance an son germe\notelexique{n'avoir soing de, expr. : « ne pas se soucier de, ne pas s'inquiéter de » (DECT), « n'avoir cure de » confondre, v.tr. : « détruire » (gl), « anéantir » (DECT) gent, s.f. : « peuple ».}\\
\verseline que li fruiz vint an sa nature\notemorphologie{se, conj. de subordination à valeur hypothétique vialt : IP3 voloir (forme seconde et picardisante de veult) faire le creante de qqn, expr. : « accomplir la volonté de qqn. » (DECT).}\\
\verseline d’anfant: plus bele criature\\
\verseline ne pot estre ne loing ne pres.\notemorphologie{mainne : IP3 mener dont le COD est quarante chevaliers et Soredamors et son fils. Attention à la polysyndète qui a du sens (voir les deux vers suivants) et mérite d'être traduite.}\\
\notelexique{v. 2364: 'nature d'enfant' exprime l'état ou la condition d'enfant. v. 2365: 'ne pot' au sens impersonnel 'il ne put'.}\notemorphologie{Tant... que : Corrélation consécutive avec indicatif. Ne pot : Négation simple d'un semi-auxiliaire suivi de l'infinitif 'estre'. Bele : Adjectif biforme (B+e).}
\modernfrench
La semence fut si longtemps en son germe que le fruit vint à sa nature d'enfant : il ne put y avoir de plus belle créature ni de loin ni de près.

\end{translation}

\end{translationsection}

\begin{translationsection}{v. 2421--2423}

\begin{translation}

\oldfrench
\verseline L’anfant apelerent Cligés.\\
\verseline Ce est Cligés an cui mimoire\notemorphologie{ices : pron. démonstratif (forme renforcée de ces sans expressivité particulière par i-) renvoyant à Soredamor et Cligés. vost : PS3 voloir faire a + inf., expr. : « être à », d'où « mériter ».}\\
\verseline fu mise an romans ceste estoire.\\
\notemorphologie{v. 2367: 'an cui mimoire', relatif 'cui' en fonction de déterminant possessif (CRA). v. 2368: 'roman' désigne la langue vulgaire par opposition au latin.}\notesyntaxe{Anfant : Substantif à deux bases (B2). Cligés : Nom propre masculin indéclinable (base en -s). Ce est : Tournure présentative avec pronom neutre 'ce'.}
\modernfrench
Ils appelèrent l'enfant Cligès. C'est Cligès en la mémoire duquel cette histoire fut mise en roman.

\end{translation}

\end{translationsection}

\begin{translationsection}{v. 2424--2427}

\begin{translation}

\oldfrench
\verseline De lui et de son vasselage,\notelexique{monter sur mer, expr. : « embarquer ».}\\
\verseline qant il iert venuz en aage\notemorphologie{orent : PS6 avoir dont le COD est boen vant nes : (nef+s voir également au vers suivant cers : cerf+s de cervus pour une variante combinatoire de même type) CSS de nef < navis « bateau, bâtiment ») Attention à tost, adv. temporel, faux ami : « rap...}\\
\verseline que il devra en pris monter,\\
\verseline m'orroiz assés dire et conter.\\
\notelexique{v. 2369: 'vasselage' désigne les qualités de chevalerie. v. 2372: Doublet synonymique 'dire et conter' (hyperonyme/terme spécialisé).}\notemorphologie{Iert/Orroiz : Formes de futur (F1). 'Iert' est la forme étymologique (erit), 'orroiz' est régulier (B+r+ez).}
\modernfrench
De lui et de sa prouesse, quand il sera venu à l'âge où il devra monter en prix, vous m'entendrez dire et raconter bien des choses.

\end{translation}

\end{translationsection}

\begin{translationsection}{v. 2428--2431}

\begin{translation}

\oldfrench
\verseline Mes an la fin an Grece avint\notemorphologie{einz que, loc. conj. temporelle : « avant que » ce cuit (cuit : IP1 cuidier) : intervention du narrateur qui donne ici, à la P1, un commentaire pristrent : PS6 prendre prendre port, expr. : « mouiller dans un port, arriver dans un port » Attention...}\\
\verseline qu’a sa fin l’empereres vint\\
\verseline qui Constantinoble tenoit.\\
\verseline Morz fu, morir le covenoit,\notemorphologie{empereres, CSS : noter le -s analogique de la première déclinaison, sujet de ert ert : indicatif impft. estre (P3 forme étymologique) i : pron. adv. lieu, renvoyant à la cité, ne pas l'oublier dans la traduction por verité : assertion constitutant...}\\
\notemorphologie{v. 2374: 'venir a sa fin' est une expression pour mourir. v. 2376: 'covenoit', imparfait de 'covenir'.}\notemorphologie{Avint qu'a sa fin... : Complétive sujet d'un verbe unipersonnel. Empereres : Substantif masculin à deux bases (CSS B1). Tennoit/Covenoit : Imparfaits réguliers (P3).}
\modernfrench
Mais à la fin il advint en Grèce que l'empereur qui tenait Constantinople vint à sa fin. Il mourut, il lui fallait mourir,

\end{translation}

\end{translationsection}

\begin{translationsection}{v. 2432--2435}

\begin{translation}

\oldfrench
\verseline qu’il ne pot son terme passer.\\
\verseline Mes einz sa mort fist amasser\\
\verseline toz les hauz barons de sa terre\\
\verseline por Alixandre anvoier querre\notemorphologie{tantost que, loc. conj. : « aussitôt que, dès que » Noter la nuance étymologique de ariver (< *ad-ripare, de ripa « rive ») Un privé est un homme de confiance, qui fait partie du premier cercle d'un seigneur un suen : il faut choisir dans la tradu...}\\
\notemorphologie{v. 2377: 'pot' est un PS3 de pooir. v. 2378: 'einz' signifie avant. 'Amasser' signifie ici rassembler (régime animé). 'Baron' est un faux ami (seigneur de premier plan).}\notesyntaxe{Ne pot... passer : Négation d'un semi-auxiliaire. Fist amasser : Périphrase actancielle d'immixtion causative. Por... querre : Infinitif de but.}
\modernfrench
car il ne put dépasser son terme. Mais avant sa mort il fit rassembler tous les hauts barons de sa terre pour envoyer chercher Alexandre

\end{translation}

\end{translationsection}

\begin{translationsection}{v. 2436--2439}

\begin{translation}

\oldfrench
\verseline an Bretaigne, ou il estoit,\\
\verseline ou molt volantiers s’ arestoit.\\
\verseline De Grece murent li message,\\
\verseline par mer acuellent lor veage,\\
\notesyntaxe{v. 2381-82: Deux relatives en 'ou' avec antécédent 'Bretaigne'. v. 2384: 'acuellent' signifie commencer.}\notemorphologie{Ou il estoit : Relative adjective introduite par 'ou'. Murent : Passé simple P6 de movoir (B5+rent). Estoit : Imparfait analogique de estre.}
\modernfrench
en Bretagne, où il était et où il séjournait très volontiers. Les messagers partirent de Grèce, ils commencent leur voyage par mer,

\end{translation}

\end{translationsection}

\begin{translationsection}{v. 2440--2443}

\begin{translation}

\oldfrench
\verseline si les a pris une tormante\\
\verseline qui lor nef et lor gent tormante.\notemorphologie{message, s.m. : « messager » attention au faux ami Il faut de traduire courtois et sage, les deux qualificatifs mélioratifs de chevalier. Est courtois celui qui a toutes les qualités pour vivre à la cour selon un haut idéal d'honneur et de morale ...}\\
\verseline En la mer furent tuit noié\\
\verseline fors un felon, un renoié,\\
\noteother{v. 2385-86: Rime riche entre le nom 'tormante' (tempête) et le verbe 'tormante' (accabler). v. 2388: 'fors' exprime l'exception. 'Felon' et 'renoié' sont quasi-synonymes.}\notemorphologie{Tuit : Pronom indéfini CSPM. Felon : Adjectif substantivé à deux bases (B2). Nef+s > nes : Effacement de la labiodentale devant s de flexion.}
\modernfrench
lorsqu'une tempête les surprit, malmenant leur nef et leurs gens. En mer ils furent tous noyés, sauf un félon, un renégat,

\end{translation}

\end{translationsection}

\begin{translationsection}{v. 2444--2447}

\begin{translation}

\oldfrench
\verseline qui amoit Alis le menor\\
\verseline plus qu’Alixandre le graignor.\\
\verseline Qant il fu de mer eschapez,\\
\verseline an Grece s’an est retornez\notemorphologie{Construire de la manière suivante : si ancessor (déterminant possessif CSMP et ancessor CSP « ancêtre ») est le sujet de avoient eüe qui a molt grant seignorie (au sens de « domaine, possession ») comme COD. Tos tans, an la cité et d'ancesserie (e...}\\
\notemorphologie{v. 2389-90: 'menor' (cadet) et 'graignor' (aîné) sont des comparatifs de supériorité au CRS de 'petit' et 'grand'.}\notesyntaxe{Menor/Graignor : Adjectifs de la classe 4 (alternance de bases), comparatifs synthétiques. Plus... que : Comparative d'inégalité elliptique.}
\modernfrench
qui aimait Alis le cadet plus qu'Alexandre l'aîné. Quand il fut échappé de la mer, il s'en est retourné en Grèce

\end{translation}

\end{translationsection}

\begin{translationsection}{v. 2448--2451}

\begin{translation}

\oldfrench
\verseline et dit qu’avoient tuit esté\\
\verseline dedanz cele mer tenpesté\\
\verseline qant de Bretaigne revenoient\notelexique{seüe : p.pass savoir la chose est développé par le contenu du vers 2451 « à savoir que ».}\\
\verseline et lor seignor en ramenoient;\\
\notelexique{v. 2396: 'lor seignor' désigne ici Alexandre (prétendument mort).}\notemorphologie{Dit qu'avoient... : Complétive COD d'un verbe de déclaration. Cele mer : Démonstratif de la série en -l (référent éloigné). Revenoient/Ramenoient : Imparfaits P6.}
\modernfrench
et dit qu'ils avaient tous été malmenés par la tempête dans cette mer alors qu'ils revenaient de Bretagne et ramenaient leur seigneur ;

\end{translation}

\end{translationsection}

\begin{translationsection}{v. 2452--2455}

\begin{translation}

\oldfrench
\verseline n’en est eschapez mes que il\\
\verseline de la tormante et del peril.\notesyntaxe{chalongier, v.tr. : « revendiquer » ne ce : avec ne coordonnant en milieu négatif et ce pron. démonstratif cataphorique annonçant le contenu du vers 2455 Le sujet de va chalongier est Acorïonde, celui de pardone Alexandre, le il de il l'a tenue re...}\\
\verseline Cil fu creüz de sa mançonge ;\\
\verseline sanz contredit et sanz chalonge\\
\notelexique{v. 2397: 'ne...mes que' expression de l'exception. v. 2399: 'cil' renvoie au renégat. 'Mançonge' est féminin en ancien français. v. 2400: 'chalonge' signifie contestation.}\notemorphologie{Mes que il : Tour exceptif à base 'mais'. Cil : Pronom démonstratif sujet (CS) marquant un changement de focus narratif. Chalonge : Substantif masculin de type 1.}
\modernfrench
personne d'autre que lui n'est échappé de la tempête et du péril. Celui-ci fut cru dans son mensonge ; sans contredit et sans contestation

\end{translation}

\end{translationsection}

\begin{translationsection}{v. 2456--2460}

\begin{translation}

\oldfrench
\verseline prenent Alis, si le coronent,\notemorphologie{el < en + le : enclise de l'article défini et de la préposition assez, adv. : qui dit le grand nombre, pas forcément la satiété en langue ancienne conjoïr, v.tr. : « accueillir avec amitié, faire fête » (gl) trueve : IP3 trover ne ne dit mot : dis...}\\
\verseline l’empire de Grece li donent.\\
\verseline Mes ne tarda mie granmant\\
\verseline qu’Alixandres certainnemant\\
\verseline sot qu’anperere estoit Alis.\notemorphologie{einçois : voir supra atant : IP3 attendre avec Acorïonde pour actant (stricto sensu sujet pronominal effacé en place 3) oie : sub.pré. P3 oïr corage, s.m. : « intention, état d'esprit » (DECT), attention au faux ami lor droit seignorage : avec sei...}\\
\notemorphologie{v. 2401: 'si' est un adverbe de liaison (sic) sans valeur hypothétique. v. 2403: 'ne tarda mie', sujet impersonnel non exprimé. v. 2405: 'sot' est un PS3 de savoir.}\notemorphologie{Ne tarda mie... que : Complétive sujet d'un verbe unipersonnel. Empire de Grece : Complément du nom (de + pays). Sot : Passé simple fort en -u (B5+t).}
\modernfrench
ils prennent Alis, le couronnent et lui donnent l'empire de Grèce. Mais il ne tarda guère qu'Alexandre sût avec certitude qu'Alis était empereur.

\end{translation}

\end{translationsection}

\begin{translationsection}{v. 2461--2463}

\begin{translation}

\oldfrench
\verseline Au roi Artus a congié pris,\\
\verseline qu’il ne voldra mie sanz guerre\\
\verseline a son frere lessier sa terre.\notelexique{finer, v. emploi intr. : « s'arrêter, cesser » nel < ne+le : enclise du PPC (Alix) et de l'adverbe négatif. ancliner, v. : « s'incliner », vient ajouter une précision en doublet quasi-synonymique à saluer.}\\
\notemorphologie{v. 2407-08: 'voldra' (futur) marque peut-être le discours indirect libre (justification d'Alexandre).}\notesyntaxe{Ne voldra mie : Négation renforcée par le forclusif 'mie'. Lessier : Infinitif en fonction de COD du verbe 'voldra'.}
\modernfrench
Il prit congé du roi Arthur, car il ne voudra en aucun cas laisser sa terre à son frère sans faire la guerre.

\end{translation}

\end{translationsection}

\begin{translationsection}{v. 2464--2467}

\begin{translation}

\oldfrench
\verseline Li rois de rien ne le destorbe,\\
\verseline einçois li dit que si grant torbe\\
\verseline en maint avoec lui de Galois,\notemorphologie{aport : IP1 aporter defors, adv.lieu : « dehors, à l'extérieur ».}\\
\verseline d’Escoz et de Cornoalois\\
\notelexique{v. 2409: 'destorber' signifie empêcher. v. 2410: 'einçois' marque une rectification/renchérissement. 'Torbe' signifie foule ou grand nombre.}\notemorphologie{De rien ne... : Négation renforcée par 'rien'. Maint : Subjonctif présent P3 (valeur d'ordre après 'dit'). Galois : Adjectif substantivé.}
\modernfrench
Le roi ne l'en empêche en rien, au contraire il lui dit d'emmener avec lui une si grande troupe de Gallois, d'Écossais et de Cornouaillais

\end{translation}

\end{translationsection}

\begin{translationsection}{v. 2468--2471}

\begin{translation}

\oldfrench
\verseline que ses frere atandre ne l’ost,\\
\verseline qant assanblee verra s’ost.\notelexique{antant, impératif de entendre : qui n'a pas ici le sens de base de « percevoir par l'ouie » mais doit être renforcé « prêter attention, écouter attentivement » demande et quiert ont pour actant tes freres, soit Alexandre (stricto sensu sujet prono...}\\
\verseline Alixandres, se lui pleüst,\\
\verseline grant force menee en eüst,\\
\notemorphologie{v. 2413: 'ost' (oser) et v. 2414 'ost' (armée) sont homonymes. v. 2415-16: Système hypothétique au subjonctif imparfait (potentiel).}\notemorphologie{Si grant... que : Corrélation consécutive. Ses frere : Sujet non marqué (type 2, CSS sans -s). Pleüst/Eüst : Subjonctifs imparfaits de verbes forts en -u (B6+ss+t).}
\modernfrench
que son frère n'ose pas l'attendre quand il verra son armée assemblée. Alexandre, s'il l'avait voulu, aurait emmené une grande force,

\end{translation}

\end{translationsection}

\begin{translationsection}{v. 2472--2475}

\begin{translation}

\oldfrench
\verseline mes n’a soing de sa gent confondre,\notemorphologie{iert : futur P3 estre soe, CSSF du caractérisant possessif : ici construit en attribut du sujet Constantinoble tenir au sens fort de « diriger ».}\\
\verseline se ses freres li vialt respondre\\
\verseline que il li face son creante.\notemorphologie{eüst : subj. impft P3 avoir à valeur de potentiel.}\\
\verseline Chevaliers an mainne quarante\\
\notemorphologie{v. 2417: 'n'avoir soing de' signifie ne pas se soucier de. 'Confondre' signifie détruire. v. 2419: 'faire le creante de qqn' signifie accomplir sa volonté.}\notemorphologie{N'a soing de... : Négation d'une locution verbale. Vialt : Indicatif présent P3 (forme picarde de 'veult'). Face : Subjonctif présent (volonté).}
\modernfrench
mais il n'a cure de détruire son peuple, si son frère veut bien accepter d'accomplir sa volonté. Il emmène quarante chevaliers

\end{translation}

\end{translationsection}

\begin{translationsection}{v. 2476--2479}

\begin{translation}

\oldfrench
\verseline et Soredamors et son fil:\notemorphologie{soi acorder, v. pron. : « trouver un accord », « se réconcilier » (DECT) si : adv. de phrase saturant la place 1 auquel on peut trouver ici une nuance consécutive rant : impératif de rendre pes, s.f. : « paix » les : IP2 de laire/layer « laisser ».}\\
\verseline ices .II. lessier ne vost il\\
\verseline car molt feisoient a amer.\\
\verseline A Sorhan monterent sor mer\noteother{biax dolz ami : apostrophe marquant le changement d'interlocuteur dans le discours direct. À traduire en évitant le calque.}\\
\notelexique{v. 2422: 'ices' est un démonstratif renforcé. v. 2423: 'faire a + inf.' signifie mériter de. v. 2424: 'monter sor mer' signifie embarquer.}\notemorphologie{Ices : Pronom démonstratif CR (référence anaphorique). Vost : Passé simple P3 de voloir (B5+t). Feisoient a amer : Périphrase modale.}
\modernfrench
ainsi que Soredamors et son fils : il ne voulut pas laisser ces deux-là car ils méritaient grandement d'être aimés. À Southampton ils s'embarquèrent sur mer

\end{translation}

\end{translationsection}

\begin{translationsection}{v. 2480--2483}

\begin{translation}

\oldfrench
\verseline au congié de tote la cort.\notemorphologie{ies : IP2 estre qui... aporté : proposition relative dont l'antécédent est le PPC représentant Acorïonde « toi qui... ».}\\
\verseline Boen vant orent, la nes s’an cort\\
\verseline assez plus tost que cers qui fuit.\noteother{Noter que l'adverbe négatif ne peut saturer la place 1.}\\
\verseline Einz que passast li mois, ce cuit,\\
\notelexique{v. 2426: 'nes' (nef+s) et 'cers' (cerf+s). 'Tost' est un faux ami (rapidement). v. 2428: 'ce cuit' est une intervention du narrateur.}\notesyntaxe{La nes s'an cort : Substantif féminin indéclinable (base en -s). Plus tost que : Comparative de différence. Einz que : Conjonction d'antériorité.}
\modernfrench
avec le congé de toute la cour. Ils eurent bon vent, le navire file plus rapidement qu'un cerf qui s'enfuit. Avant que le mois ne passât, je crois,

\end{translation}

\end{translationsection}

\begin{translationsection}{v. 2484--2487}

\begin{translation}

\oldfrench
\verseline pristrent devant Athenes port,\notemorphologie{croi : IP1 croire as < a+les : enclise de l'article défini et de la préposition a.}\\
\verseline une cité molt riche et fort.\notemorphologie{vis : (vif+s, variante combinatoire déjà notée), CSSM de vif « vivant » jel < je+le : enclise du PPS et du PPC nel< ne+le : enclise de l'adverbe négatif et du PPC cresrai : forme inhabituelle du futur P1 de croire.}\\
\verseline L’empereres an la cité\\
\verseline ert a sejor por verité\notemorphologie{piece a, loc. adv. : « il y a longtemps », « depuis un certain temps » poise : IP3 peser dont le sujet est le pronom démonstratif ce.}\\
\notemorphologie{v. 2430: 'riche' est un faux ami (puissant). v. 2432: 'ert' est l'imparfait étymologique de estre. v. 2433: 'por verité' est un commentaire (assurément).}\notemorphologie{Pristrent : Passé simple P6 de prendre. Riche et fort : Adjectifs épicènes (classe 2 et 3). Ert : Imparfait P3 étymologique (erat).}
\modernfrench
ils arrivèrent au port d'Athènes, une cité très puissante et forte. L'empereur se trouvait assurément dans la cité en séjour

\end{translation}

\end{translationsection}

\begin{translationsection}{v. 2488--2491}

\begin{translation}

\oldfrench
\verseline et s’1 avoit grant assanblee\notemorphologie{diës : sub.présent P3 dire.}\\
\verseline des hauz barons de la contree.\\
\verseline Tantost qu’il furent arivé,\notemorphologie{covient : forme impersonnelle li : PPC renvoyant à Alexandre doigne : sub.présent P1 donner assez, adv. : au sens de « un grand nombre de ».}\\
\verseline Alixandres un suen privé\\
\noteother{v. 2435: 'tantost que' (aussitôt que). 'Ariver' (ad-ripare, venir à la rive). v. 2436: 'privé' est un homme de confiance. 'Un suen' (un sien).}\notemorphologie{Tantost que : Conjonction de concomitance. Des hauz barons : Article contracté (de+les). Furent : Passé simple de estre.}
\modernfrench
et il y avait là une grande assemblée des hauts barons de la contrée. Aussitôt qu'ils furent arrivés, Alexandre envoie un de ses familiers

\end{translation}

\end{translationsection}

\begin{translationsection}{v. 2492--2495}

\begin{translation}

\oldfrench
\verseline envoie an la cité savoir\notemorphologie{fos : (fol+s avec vocalisation non marquée dans la graphie), adj. CSSM, « fou » se / s' : conj. de subordination à valeur hypothétique iert : futur P3 estre ne ja : négation renforcée à valeur absolue « ne... pas du tout, ne... absolument pas ».}\\
\verseline se recet i porroit avoir\\
\verseline ou s’il li voldront contredire\notemorphologie{de, préposition : qui a ici sa valeur latine (de + abl.) « au sujet de..., quant à » nus : (nul+s CSSM) sujet de iert tenanz, « sera détenteur » ne... ja : négatif renforcée à valeur temporelle « ne... jamais ».}\\
\verseline qu’il ne soit lor droituriers sire.\\
\notelexique{v. 2437: 'avoir recet' signifie être bien accueilli. v. 2440: 'droiturier sire' (seigneur légitime et juste).}\notemorphologie{Savoir se... ou se... : Interrogative totale double. Voldront : Futur P6. Qu'il ne soit : Subjonctif après verbe de contestation.}
\modernfrench
dans la cité pour savoir s'il pourrait y être bien accueilli ou s'ils voudront lui contester d'être leur seigneur légitime.

\end{translation}

\end{translationsection}

\begin{translationsection}{v. 2496--2499}

\begin{translation}

\oldfrench
\verseline De ceste chose fu messages\notemorphologie{cil : pronom démonstratif renvoyant à Acorïonde ot : IP3 oïr la responsse l'empereor : complément déterminatif du nom construit en CRA avenant, p.présent : « convenable ». \end{document}}\\
\verseline uns chevaliers cortois et sages\\
\verseline qu’an apeloit Acorïonde,\\
\verseline riches d’avoir et de faconde,\\
\notelexique{v. 2441: 'message' est un faux ami (messager). v. 2442: 'cortois' et 'sage' sont des qualificatifs mélioratifs.}\notemorphologie{Ceste chose : Démonstratif de la série en -st (cotexte immédiat). Apeloit : Imparfait P3. Cortois et sages : Adjectifs (classe 2 et 1).}
\modernfrench
Un chevalier courtois et avisé nommé Acorionde, riche en biens et en éloquence, fut le messager de cette affaire ;

\end{translation}

\end{translationsection}

\begin{translationsection}{v. 2500--2503}

\begin{translation}

\oldfrench
\verseline et s’estoit molt bien del païs\\
\verseline car d’ Athenes estoit naïs.\\
\verseline An la cité d’ancesserie\\
\verseline avoient molt grant seignorie\\
\notemorphologie{v. 2445: 'estre bien de qqn' (être en bons termes). v. 2446: 'naïs' (natif). v. 2449: 'd'ancesserie' (depuis fort longtemps).}\notemorphologie{Naïs : Adjectif biforme (B+s, base naïf). Seignorie : Substantif féminin de type 1. Païs : Substantif masculin indéclinable (base en -s).}
\modernfrench
il était très bien vu du pays car il était natif d'Athènes. Dans la cité, depuis fort longtemps, ses ancêtres avaient possédé un très grand domaine

\end{translation}

\end{translationsection}

\begin{translationsection}{v. 2504--2507}

\begin{translation}

\oldfrench
\verseline tos tans si ancessor eüe.\\
\verseline Qant il a la chose seüe\\
\verseline qu’an la vile estoit l’emperere,\\
\verseline de par Alixandre son frere\\
\notesyntaxe{v. 2449: 'si ancessor' (ses ancêtres). v. 2450-51: 'la chose seüe' développée par la complétive 'qu'an la vile...'.}\notemorphologie{Si ancessor : Déterminant possessif pluriel. Seüe : Participe passé de savoir. Qu'an la vile... : Complétive explicative.}
\modernfrench
qu'ils avaient toujours tenu. Quand il apprit que l'empereur était dans la ville, au nom de son frère Alexandre,

\end{translation}

\end{translationsection}

\begin{translationsection}{v. 2508--2510}

\begin{translation}

\oldfrench
\verseline li va chalongier la corone\\
\verseline ne ce mie ne li pardone\\
\verseline qu’il l’a tenue contre droit.\\
\noteother{v. 2453: 'chalongier' (revendiquer). v. 2454: 'ne ce' (coordonnant + démonstratif cataphorique).}\notesyntaxe{Li va chalongier : Infinitif de progrédience (mouvement). Ne... mie : Négation renforcée par 'mie'. Ce... qu'il l'a tenue : Complétive développant un démonstratif.}
\modernfrench
il va lui revendiquer la couronne et ne lui pardonne nullement de l'avoir tenue contre le droit.

\end{translation}

\end{translationsection}

\begin{translationsection}{v. 2511--2514}

\begin{translation}

\oldfrench
\verseline El palés est venuz tot droit\\
\verseline et trueve assez qui le conjot,\\
\verseline mes ne respont ne ne dit mot\\
\verseline a nul home qui le conjoie,\\
\noteother{v. 2456: 'el' (en+le). 'Assez' exprime le grand nombre. v. 2457: 'conjoïr' (accueillir avec amitié). v. 2459: 'ne ne dit mot' (coordonnant + négation).}\notemorphologie{El palés : Article contracté. Ne dit mot : Négation renforcée par un forclusif occasionnel ('mot'). A nul home qui... : Relative au subjonctif (antécédent nié).}
\modernfrench
Il est arrivé tout droit au palais et trouve bien des gens qui lui font fête, mais il ne répond ni ne dit mot à personne qui l'accueille,

\end{translation}

\end{translationsection}

\begin{translationsection}{v. 2515--2517}

\begin{translation}

\oldfrench
\verseline einçois atant tant que il oie\\
\verseline quel volanté et quel corage\\
\verseline il ont vers lor droit seignorage.\\
\noteother{v. 2460: 'atant' (attendre). v. 2461: 'corage' (intention). v. 2462: 'seignorage' (autorité, ici le seigneur).}\notemorphologie{Einçois : Adverbe de rectification. Tant que il oie : Circonstancielle temporelle (attente) au subjonctif. Quel volanté... : Interrogative partielle.}
\modernfrench
au contraire il attend d'entendre quelles sont les intentions et l'état d'esprit qu'ils ont envers leur seigneur légitime.

\end{translation}

\end{translationsection}

\begin{translationsection}{v. 2518--2520}

\begin{translation}

\oldfrench
\verseline Jusqu’a l’empereor ne fine; il nel salue ne ancline\\
\verseline ne empereor ne l’apele :\\
\verseline «Alis, fet il, une novele\\
\noteother{v. 2463: 'finer' (s'arrêter). v. 2464: 'nel' (ne+le). 'Ancliner' (s'incliner).}\notemorphologie{Il nel salue ne ancline... : Négation totale sans renforcement. Salue/Ancline/Apele : Présents de l'indicatif P3 (classe 1).}
\modernfrench
Il ne s'arrête qu'arrivé devant l'empereur ; il ne le salue ni ne s'incline devant lui et ne l'appelle pas empereur : « Alis, dit-il, j'apporte une nouvelle

\end{translation}

\end{translationsection}

\begin{translationsection}{v. 2521--2523}

\begin{translation}

\oldfrench
\verseline de par Alixandre t’aport,\\
\verseline qui la defors est a ce port.\\
\verseline Antant que tes freres te mande:\\
\notemorphologie{v. 2467: 'aport' (P1 aporter). v. 2468: 'defors' (dehors). v. 2469: 'antant' (prêter attention, écouter).}\notemorphologie{Ce port : Démonstratif déictique. Antant que... : Interrogative partielle (objet de l'ordre). Aport : Présent indicatif P1 (classe 1).}
\modernfrench
de la part d'Alexandre qui est là-bas au port. Écoute ce que ton frère te mande :

\end{translation}

\end{translationsection}

\begin{translationsection}{v. 2524--2526}

\begin{translation}

\oldfrench
\verseline la soe chose te demande\\
\verseline ne contre reison rien ne quiert.\\
\verseline Soe doit bien estre et soe iert\\
\notemorphologie{v. 2471: 'la soe chose' (son bien). v. 2473: 'soe' (caractérisant possessif attribut). 'Iert' (futur P3 estre).}\notemorphologie{Rien ne quiert : Négation renforcée par 'rien'. Iert : Futur étymologique (erit). Quiert : Présent de l'indicatif à deux bases (B2, infinitif querre).}
\modernfrench
il te demande son bien et ne requiert rien contre la raison. Elle doit bien être sienne et sienne sera

\end{translation}

\end{translationsection}

\begin{translationsection}{v. 2527--2529}

\begin{translation}

\oldfrench
\verseline Costantinoble que tu tiens.\\
\verseline Ce ne seroit reisons ne biens\\
\verseline qu’antre vos .Il. eüst descorde.\\
\noteother{v. 2473: 'tenir' (diriger).}\notemorphologie{Que tu tiens : Relative adjective. Ce ne seroit... : Tournure présentative avec 'ce' neutre. Eüst : Subjonctif imparfait de verbe fort (B6+ss+t).}
\modernfrench
Constantinople que tu diriges. Ce ne serait ni raisonnable ni bien qu'il y eût entre vous deux de discorde.

\end{translation}

\end{translationsection}

\begin{translationsection}{v. 2530--2532}

\begin{translation}

\oldfrench
\verseline Par mon consoil a lui t’acorde,\\
\verseline si li rant la corone an pes,\\
\verseline car bien est droiz que tu li les.»\\
\notemorphologie{v. 2476: 'soi acorder' (se réconcilier). v. 2477: 'si' (adverbe avec nuance consécutive). 'Pes' (paix). v. 2478: 'les' (P2 laire, laisser).}\notemorphologie{Acorde/Rant : Impératifs. Les : Présent indicatif P2. Que tu li les : Complétive (sujet réel de 'est droiz') au subjonctif.}
\modernfrench
Par mon conseil réconcilie-toi avec lui, ainsi rends-lui la couronne en paix, car il est juste que tu la lui laisses. »

\end{translation}

\end{translationsection}

\begin{translationsection}{v. 2533--2535}

\begin{translation}

\oldfrench
\verseline Alis respont: «Biax dolz amis,\\
\verseline de folie t’ies antremis,\\
\verseline qui cest message as aporté.\\
\notesyntaxe{v. 2479: 'biax dolz ami' (apostrophe). v. 2481: 'qui... aporté' (relative renvoyant à 'toi').}\notemorphologie{Biax dolz amis : Apostrophe sans article. Cest message : Démonstratif anaphorique. T'ies : Présent P2 de estre.}
\modernfrench
Alis répond : « Beau cher ami, tu t'es chargé d'une folie, toi qui as apporté ce message.

\end{translation}

\end{translationsection}

\begin{translationsection}{v. 2536--2538}

\begin{translation}

\oldfrench
\verseline Ne m'as de rien reconforté,\\
\verseline car bien sai que mes frere est morz.\\
\verseline Ne croi pas que il soit as porz.\\
\noteother{v. 2484: 'as' (a+les).}\notemorphologie{Ne... de rien : Négation renforcée. Ne croi pas que... : Principale négative régissant le subjonctif ('soit'). Sai : Présent P1 (B3).}
\modernfrench
Tu ne m'as en rien réconforté, car je sais bien que mon frère est mort. Je ne crois pas qu'il soit au port.

\end{translation}

\end{translationsection}

\begin{translationsection}{v. 2539--2541}

\begin{translation}

\oldfrench
\verseline S'il estoit vis et jel savoie,\\
\verseline ja nel cresrai tant que jel voie.\\
\verseline Morz est piece a, ce poise moi;\\
\notemorphologie{v. 2485: 'vis' (vif+s). v. 2486: 'jel' (je+le). 'Nel' (ne+le). 'Cresrai' (futur P1 croire). v. 2487: 'piece a' (depuis longtemps). 'Poise' (peser).}\notemorphologie{S'il estoit... : Système hypothétique (imparfait/futur). Ja nel cresrai : Négation renforcée par 'ja'. Tant que jel voie : Temporelle au subjonctif.}
\modernfrench
S'il était vivant et que je le lusse, jamais je ne le croirai tant que je ne le verrai pas. Il est mort depuis longtemps, cela m'afflige ;

\end{translation}

\end{translationsection}

\begin{translationsection}{v. 2542--2544}

\begin{translation}

\oldfrench
\verseline rien que tu diës je ne croi.\\
\verseline Et s’il est vis, por coi ne vient ?\\
\verseline Ja redoter ne li covient\\
\notemorphologie{v. 2488: 'diës' (subjonctif P3 dire). v. 2490: 'covient' (forme impersonnelle).}\notemorphologie{Rien que... : Relative au subjonctif (antécédent indéterminé). Por coi ne vient ? : Interrogation partielle négative. Ja... ne : Négation absolue.}
\modernfrench
je ne crois rien de ce que tu dis. Et s'il est vivant, pourquoi ne vient-il pas ? Il ne lui convient en aucun cas de craindre

\end{translation}

\end{translationsection}

\begin{translationsection}{v. 2545--2547}

\begin{translation}

\oldfrench
\verseline que assez terre ne li doigne ;\\
\verseline Fos est, se il de moi s’esloigne,\\
\verseline et s’il me sert, ja n’en iert pire.\\
\notemorphologie{v. 2491: 'doigne' (subjonctif P1 donner). 'Assez' (grand nombre de). v. 2492: 'fos' (fol+s). v. 2493: 'iert' (futur P3 estre).}\notemorphologie{Que... ne li doigne : Complétive après verbe de crainte. Ja n'en iert pire : Négation renforcée temporelle. Fos : Adjectif (classe 1, accident phonétique devant -s).}
\modernfrench
que je ne lui donne maintes terres ; il est fou s'il s'éloigne de moi, et s'il me sert, il ne s'en portera jamais plus mal.

\end{translation}

\end{translationsection}

\begin{translationsection}{v. 2548--2550}

\begin{translation}

\oldfrench
\verseline De la corone et de l’empire\\
\verseline n’iert ja nus contre moi tenanz.»\\
\verseline Cil ot que n’est pas avenanz\\
\noteother{v. 2494: 'de' (quant à). v. 2495: 'nus' (nul+s). 'Tenanz' (détenteur). v. 2496: 'cil' renvoie à Acorïonde. 'Avenant' (convenable).}\notesyntaxe{N'iert ja nus... : Négation renforcée. Tenanz/Avenanz : Adjectifs verbaux (classe 3). Cil : Pronom démonstratif de la série en -l.}
\modernfrench
Quant à la couronne et l'empire, jamais personne ne les détiendra contre moi. » Celui-ci entend que n'est pas convenable

\end{translation}

\end{translationsection}

\end{document}
