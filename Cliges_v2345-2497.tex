\documentclass[12pt,a4paper]{article}
\usepackage[utf8]{inputenc}
\usepackage[T1]{fontenc}
\usepackage[french]{babel}
\usepackage{tradaf}
\usepackage{hyperref}

\title{Notes de traduction\\
\large Chrétien de Troyes, \textit{Cligés}\\
v. 2345--2497}
\author{D'après Valérie Naudet}
\date{\today}

\begin{document}

\maketitle

\section*{Référence}

Édition bilingue de Laurence Harf-Lancner, Paris, Champion, Champion Classiques Moyen Âge, 2006.

\section*{Abréviations}

Quelques abréviations, outre celles courantes et non relevées dont on use habituellement en grammaire ou lexicologie : CS/CR (cas sujet/cas régime), F/M (féminin/masculin), S/P (singulier/pluriel), le tout pouvant être combiné (CSFP : cas sujet féminin pluriel), CRA cas régime absolu ; PS passé simple, IP, présent de l'indicatif, suivis d'un chiffre renvoyant à la personne grammaticale ; PPS / PPC pronom personnel sujet ou complément.

Les mentions suivantes peuvent également être trouvées : \textit{gl} indique que la glose a été trouvée dans le glossaire de notre édition au programme, \textit{DECT} dans le dictionnaire électronique de Chrétien de Troyes (\url{http://zeus.atilf.fr/dect/}).

\section*{Éditions consultées}

\begin{itemize}
  \item Chrétien de Troyes, \textit{Cligés}, éd. et trad. Philippe Walter, dans Chrétien de Troyes, \textit{Œuvres complètes}, Daniel Poirion dir., Paris, Gallimard, « Bibliothèque de la Pléiade », 1994.
  \item Chrétien de Troyes, \textit{Cligés}, éd. et trad. Charles Méla et Olivier Collet, dans Chrétien de Troyes, \textit{Romans}, Paris, Le Livre de Poche, « La Pochothèque », 1994.
\end{itemize}

\section*{Notes détaillées}

\subsection*{v. 2345}

\textbf{\textit{un jor}} : renforcer dans la traduction la valeur cardinale de l'article indéfini, proche ici de son étymon latin, le cardinal \textit{unus, unam, unum}.

\subsection*{v. 2346}

\textbf{\textit{enor}}, s.m. : « marque de respect et d'estime, faveur » (DECT).

\subsection*{v. 2349--2350}

\textbf{\textit{del chastel}} et \textbf{\textit{de ce que}} : donner à la préposition \textit{de} son sens latin « au sujet de ».

\subsection*{v. 2350--2353}

\textbf{\textit{li rois Artus}} est le sujet de \textit{promist} dont le COD est la complétive (mise dans l'ordre des mots usuel aujourd'hui) \textit{qu'il li donroit le meillor reiaume de Gales quant sa guerre avroit finee}.

\subsection*{v. 2354}

\textbf{\textit{le jor}} : l'article défini a ici une valeur forte, démonstrative (Claude Buridant, \textit{Grammaire nouvelle de l'ancien français}, Paris, Sedes, 2000, p. 120), qu'il faut impérativement rendre dans la traduction. Il s'agit d'un souvenir de son étymon latin le démonstratif, \textit{ille}, \textit{illa}, \textit{illud}.

Attention à \textbf{\textit{sale}}, s.f., faux ami qui désigne aussi bien une pièce, souvent importante voire principale, dans le château (choix de glose du DECT) que le palais ou le château lui-même (choix de traduction de Laurence Harf-Lancner).

\subsection*{v. 2357}

\textbf{\textit{graindre}} : comparatif de supériorité CSFS de \textit{grand}.

\textbf{\textit{de ce que}} : voir supra à propos de la préposition \textit{de}.

\textbf{\textit{Fierce}}, s.f. : « reine du jeu d'échec »

\textbf{\textit{Fu fierce}} et \textbf{\textit{fu rois}} : noter la valeur descriptive du PS en ancien français, impossible à garder en l'état dans une traduction $\Rightarrow$ passer à l'imparfait.

\subsection*{v. 2359--2364}

Selon le DECT, la \textbf{\textit{grainne}} renvoie à ce qui, chez la femme, correspond au sperme masculin. C'est la rencontre des deux qui permet la germination d'où le fruit, l'enfant, naîtra. Toutefois, les trois traductions consultées, y compris celle de Laurence Harf-Lancner, donnent toutes quelque chose d'un peu différent : \textit{semance} et \textit{grainne} sont deux principes masculins que l'utérus de la femme accueille pour les façonner et les porter à terme jusqu'à la naissance ; la syntaxe invite à construire \textit{semance} et \textit{grainne} comme les compléments de l'adjectif \textit{plainne}.

\textbf{\textit{Estre en son germe}}, expr. : « germer »

\textbf{\textit{nature d'enfant}}, expr. : « état d'enfant, condition d'enfant ».

\subsection*{v. 2367}

\textbf{\textit{Cligés en cui memoire}} : CRA construit avec le pronom relatif \textit{cui} ayant Cligés pour antécédent placé, comme attendu dans ce cas, devant le nom qu'il détermine.

\subsection*{v. 2368}

Prendre \textbf{\textit{roman}}, s.m., dans son sens premier de langue vulgaire par opposition au latin, langue savante (\textit{gl}).

\subsection*{v. 2369}

Le \textbf{\textit{vasselage}}, s.m., désigne l'ensemble des qualités qui font d'un homme un bon vassal pour son seigneur, soit essentiellement la prouesse aux armes, la chevalerie, le courage, la bravoure.

\subsection*{v. 2370 et v. 2372}

\textbf{\textit{iert}} : futur P3 \textit{estre}

\textbf{\textit{orroiz}} : futur P3 \textit{oïr}

\subsection*{v. 2372}

\textbf{\textit{dire}} et \textbf{\textit{conter}} : le roman présente un doublet de termes quasi synonymes qu'il convient impérativement de traduire dans le cadre d'une version d'agrégation en mettant en évidence la relation qui les unit, ici celle d'un hyperonyme \textit{dire} par rapport à un terme plus spécialisé \textit{conter}.

\subsection*{v. 2374}

\textbf{\textit{venir a sa fin}}, expr. : « mourir ».

\subsection*{v. 2377}

\textbf{\textit{pot}} : PS3 \textit{pooir}.

\subsection*{v. 2378--2380}

\textbf{\textit{einz}}, préposition de temps : « avant »

\textbf{\textit{amasser}}, v.tr. : peut avoir en ancien français un régime animé humain, il prend lors le sens de « rassembler, réunir »

\textbf{\textit{baron}} (B1 \textit{ber}, B2 \textit{baron}, s.m. de la 3\textsuperscript{e} déclinaison) : faux ami. Si \textit{baron} aujourd'hui est l'un des plus petits titres de noblesse, il désigne au Moyen Âge une personne de premier plan dans l'entourage politique d'un seigneur. Calque impossible.

\subsection*{v. 2381--2382}

Les deux relatives introduites par \textit{ou} ont toutes deux pour antécédent \textit{Bretaigne} et c'est à Alexandre que renvoie le sujet de leurs verbes. La construction juxtaposée est aujourd'hui peu heureuse $\Rightarrow$ rétablir un lien de coordination entre les deux subordonnées.

\subsection*{v. 2383--2384}

\textbf{\textit{murent}} : PS6 \textit{movoir}

\textbf{\textit{acuellent}} : IP6 \textit{accuellir} au sens de « commencer ».

\subsection*{v. 2385--2386}

\textbf{\textit{tormante}}, s.f. : « tempête » à la rime avec \textit{tormante} IP3 de \textit{tormanter}, v.tr., « accabler, malmener ».

\subsection*{v. 2387}

\textbf{\textit{tuit}} : pron. indéfini, CSPM de \textit{tout}.

\subsection*{v. 2388}

\textbf{\textit{fors}}, prep. : expression de l'exception.

\textbf{\textit{Felon}}, s.m., CSS : entre en relation de quasi synonymie avec \textit{renoié} pour désigner un homme qui trahit sa parole et son seigneur.

\subsection*{v. 2389--2390}

\textbf{\textit{menor}} et \textbf{\textit{graignor}} : deux comparatifs de supériorité au CRS, respectivement de \textit{petit} et de \textit{grand}.

\subsection*{v. 2392--2393}

\textbf{\textit{s'an est retornez}} et \textbf{\textit{dit}} ont le parjure pour sujet.

\subsection*{v. 2396}

\textbf{\textit{lor seigneur}} : désigne ici Alexandre, soi disant mort avec le reste de l'équipage et des passagers dans le naufrage dû à la tempête.

\subsection*{v. 2397}

\textbf{\textit{ne\ldots mes que}} : expression de l'exception ; \textit{il} renvoie au renégat, la forme prédicative et tonique du PPS étant explicable par le caractère elliptique de la proposition exceptive.

\subsection*{v. 2399}

\textbf{\textit{cil}}, sujet de \textit{fu creüz}, a pour référent le renégat ; noter le féminin de \textit{mensonge} en langue médiévale.

\subsection*{v. 2400}

\textbf{\textit{chalonge}}, s.m. : « contestation » (\textit{gl}). C'est de la même racine morphologique que descend le \textit{challenge} que l'anglais nous a renvoyé.

\subsection*{v. 2401}

\textbf{\textit{si}} : adverbe de phrase dont la principale fonction est la saturation de la place 1 de la proposition et le lien avec ce qui précède. Descendant de \textit{sic} latin, il n'a aucune valeur hypothétique (\textit{i.e.} traduction par \textit{si} hypothétique $\Rightarrow$ contresens grave). Sa traduction dépend du contexte : parfois il est intraduisible, parfois un lien logique apparaît clairement « pourtant » ou « alors » ou « ainsi »\ldots

\subsection*{v. 2403}

\textbf{\textit{ne tarda mie}} : le sujet impersonnel n'est pas exprimé $\Rightarrow$ à rétablir.

\subsection*{v. 2405}

\textbf{\textit{sot}} : PS3 \textit{savoir}.

\subsection*{v. 2407--2408}

Rétablir un ordre des mots fluide et conforme à la phrase moderne, éviter à tout prix le calque, syntaxique et/ou lexical dans des vers qui sont relativement transparents quant à leur sens. Le futur \textit{voldra} (P3 \textit{voloir}) peut poser problème dans la concordance des temps. Il peut s'agir ici d'une marque de discours indirect libre, une trace des propos tenus par Alexandre à Arthur lors de sa demande de congé pour justifier cette dernière.

\subsection*{v. 2409}

\textbf{\textit{destorber}}, v.tr. : « détourner, empêcher » (\textit{gl}).

\subsection*{v. 2410--2414}

\textbf{\textit{einçois}}, adv. : marquant une rectification et un renchérissement : il ouvre un énoncé qui, renchérissant sur le précédent, ne s'oppose à lui que sur la formulation : dire que le roi n'entrave en rien le projet d'Alexandre va dans le même sens que proposer des troupes (\textit{torbe}, s.f. « foule » \textit{gl}, « un grand nombre ») pour accompagner le héros.

Construire ainsi : du verbe \textit{dit} (v. 2410), dépend une complétive \textit{si grant torbe\ldots verra s'ost}. Le subjonctif \textit{maint} (sub.pré P3 \textit{mener}) implique une nuance d'ordre dans le verbe de parole, \textit{dit}, du roi. À traduire. Par ailleurs la complétive est elle-même une phrase complexe dont la principale est \textit{en maint} [sujet Alexandre] \textit{aveoc lui grant torbe de Galois, d'Escoz et de Cornoalois}. Cette principale est accompagnée qu'une subordonnée circonstancielle de conséquence traduite par une corrélation \textit{si} (v. 2410)\ldots\textit{que} (v. 2413) dont le subjonctif (\textit{ost} sub.pré P3 \textit{oser}) s'explique par celui du verbe de la principale (\textit{maint}). Enfin une subordonnée circonstancielle de temps clôt la séquence (v. 2414) dans laquelle \textit{ost} est un s.f. « armée ».

\subsection*{v. 2415--2416}

\textbf{\textit{se lui pleüst //, grant force menee en eüst}} : système hypothétique au subjonctif imparfait (\textit{pleüst} et \textit{eüst}, respectivement Sub.impft P3 de \textit{plaire} et \textit{avoir}) à valeur de potentiel.

\subsection*{v. 2417}

\textbf{\textit{n'avoir soing de}}, expr. : « ne pas se soucier de, ne pas s'inquiéter de » (DECT), « n'avoir cure de »

\textbf{\textit{confondre}}, v.tr. : « détruire » (\textit{gl}), « anéantir » (DECT)

\textbf{\textit{gent}}, s.f. : « peuple ».

\subsection*{v. 2418--2419}

\textbf{\textit{se}}, conj. de subordination à valeur hypothétique

\textbf{\textit{vialt}} : IP3 \textit{voloir} (forme seconde et picardisante de \textit{veult})

\textbf{\textit{faire le creante de qqn}}, expr. : « accomplir la volonté de qqn. » (DECT).

\subsection*{v. 2420--2421}

\textbf{\textit{mainne}} : IP3 \textit{mener} dont le COD est \textit{quarante chevaliers et Soredamors et son fils}. Attention à la polysyndète qui a du sens (voir les deux vers suivants) et mérite d'être traduite.

\subsection*{v. 2422--2423}

\textbf{\textit{ices}} : pron. démonstratif (forme renforcée de \textit{ces} sans expressivité particulière par \textit{i}-) renvoyant à Soredamor et Cligés.

\textbf{\textit{vost}} : PS3 \textit{voloir}

\textbf{\textit{faire a + inf.}}, expr. : « être à », d'où « mériter ».

\subsection*{v. 2424}

\textbf{\textit{monter sur mer}}, expr. : « embarquer ».

\subsection*{v. 2425--2426}

\textbf{\textit{orent}} : PS6 \textit{avoir} dont le COD est \textit{boen vant}

\textbf{\textit{nes}} : (\textit{nef+s} voir également au vers suivant \textit{cers} : \textit{cerf+s} de \textit{cervus} pour une variante combinatoire de même type) CSS de \textit{nef} < \textit{navis} « bateau, bâtiment »)

Attention à \textbf{\textit{tost}}, adv. temporel, faux ami : « rapidement, vite ».

\subsection*{v. 2428--2430}

\textbf{\textit{einz que}}, loc. conj. temporelle : « avant que »

\textbf{\textit{ce cuit}} (\textit{cuit} : IP1 \textit{cuidier}) : intervention du narrateur qui donne ici, à la P1, un commentaire

\textbf{\textit{pristrent}} : PS6 \textit{prendre}

\textbf{\textit{prendre port}}, expr. : « mouiller dans un port, arriver dans un port »

Attention à \textbf{\textit{riche}}, adj., faux ami : « puissant », notamment en doublet quasi synonymique avec \textit{fort}. Ne pas éluder l'un des deux adjectifs dans votre traduction.

\subsection*{v. 2431--2434}

\textbf{\textit{empereres}}, CSS : noter le \textit{-s} analogique de la première déclinaison, sujet de \textit{ert}

\textbf{\textit{ert}} : indicatif impft. \textit{estre} (P3 forme étymologique)

\textbf{\textit{i}} : pron. adv. lieu, renvoyant à \textit{la cité}, ne pas l'oublier dans la traduction

\textbf{\textit{por verité}} : assertion constitutant un commentaire du narrateur qu'il faut impérativement traduire « à vrai dire, à la vérité, assurément ».

Attention à \textbf{\textit{barons}}, voir supra.

\subsection*{v. 2435--2440}

\textbf{\textit{tantost que}}, loc. conj. : « aussitôt que, dès que »

Noter la nuance étymologique de \textbf{\textit{ariver}} (< *\textit{ad-ripare}, de \textit{ripa} « rive »)

Un \textbf{\textit{privé}} est un homme de confiance, qui fait partie du premier cercle d'un seigneur

\textbf{\textit{un suen}} : il faut choisir dans la traduction de privilégier soit le caractérisant possessif soit l'article indéfini, tout dépend du contexte, ici les deux sont valables

\textbf{\textit{recet}}, s.m. : « abri, refuge » (\textit{gl}), mais \textit{avoir recet} « avoir bon accueil » (DECT) qui semble mieux ici et conforme à la traduction proposée par l'éditrice

\textbf{\textit{il li voldront}} : \textit{il}, PPSP désignant les habitants d'Athènes

\textbf{\textit{il ne soit}} : \textit{il} renvoie à Alexandre

Un \textbf{\textit{droiturier seigneur}} est un seigneur qui à la fois occupe cette place de seigneur de manière légitime et conformément à la coutume et qui également exerce équitablement la justice. Ici Alexandre s'inquiète de savoir, si conformément au droit de primogéniture, il sera reconnu comme seigneur légitime d'Athènes au détriment de son frère cadet Alix.

\subsection*{v. 2441--2446}

\textbf{\textit{message}}, s.m. : « messager » attention au faux ami

Il faut de traduire \textbf{\textit{courtois}} et \textbf{\textit{sage}}, les deux qualificatifs mélioratifs de \textit{chevalier}. Est \textit{courtois} celui qui a toutes les qualités pour vivre à la cour selon un haut idéal d'honneur et de morale ; est \textit{sage} celui qui est cultivé, qui a les qualités requises pour mener une mission de manière réfléchie, « avisé ».

\textbf{\textit{estre bien de qqn}}, expr. : « être en bons termes avec », d'où « être bien accueilli »

\textbf{\textit{naïs}}, CCSM de \textit{naïf} : « natif ».

\subsection*{v. 2447--2449}

Construire de la manière suivante : \textbf{\textit{si ancessor}} (déterminant possessif CSMP et \textit{ancessor} CSP « ancêtre ») est le sujet de \textit{avoient eüe} qui a \textit{molt grant seignorie} (au sens de « domaine, possession ») comme COD. \textit{Tos tans}, \textit{an la cité} et \textit{d'ancesserie} (expr. « depuis fort longtemps » DECT) sont des compléments circonstanciels de temps et de lieu.

\subsection*{v. 2450--2451}

\textbf{\textit{seüe}} : p.pass \textit{savoir}

\textbf{\textit{la chose}} est développé par le contenu du vers 2451 « à savoir que ».

\subsection*{v. 2453--2455}

\textbf{\textit{chalongier}}, v.tr. : « revendiquer »

\textbf{\textit{ne ce}} : avec \textit{ne} coordonnant en milieu négatif et \textit{ce} pron. démonstratif cataphorique annonçant le contenu du vers 2455

Le sujet de \textit{va chalongier} est Acorïonde, celui de \textit{pardone} Alexandre, le \textit{il} de \textit{il l'a tenue} renvoie à Alix.

\subsection*{v. 2456--2459}

\textbf{\textit{el < en + le}} : enclise de l'article défini et de la préposition

\textbf{\textit{assez}}, adv. : qui dit le grand nombre, pas forcément la satiété en langue ancienne

\textbf{\textit{conjoïr}}, v.tr. : « accueillir avec amitié, faire fête » (\textit{gl})

\textbf{\textit{trueve}} : IP3 \textit{trover}

\textbf{\textit{ne ne dit mot}} : distinguer le premier \textit{ne} coordonnant en milieu négatif et le second adverbe de négation portant sur \textit{dit}.

\subsection*{v. 2460--2462}

\textbf{\textit{einçois}} : voir supra

\textbf{\textit{atant}} : IP3 \textit{attendre} avec Acorïonde pour actant (\textit{stricto sensu} sujet pronominal effacé en place 3)

\textbf{\textit{oie}} : sub.pré. P3 \textit{oïr}

\textbf{\textit{corage}}, s.m. : « intention, état d'esprit » (DECT), attention au faux ami

\textbf{\textit{lor droit seignorage}} : avec \textit{seignorage}, s.m., « autorité du seigneur » d'où par métonymie « seigneur » (DECT) et \textit{droit}, adj., « légitime, conformément au droit ».

\subsection*{v. 2463--2465}

\textbf{\textit{finer}}, v. emploi intr. : « s'arrêter, cesser »

\textbf{\textit{nel < ne+le}} : enclise du PPC (Alix) et de l'adverbe négatif.

\textbf{\textit{ancliner}}, v. : « s'incliner », vient ajouter une précision en doublet quasi-synonymique à \textit{saluer}.

\subsection*{v. 2466--2468}

\textbf{\textit{aport}} : IP1 \textit{aporter}

\textbf{\textit{defors}}, adv.lieu : « dehors, à l'extérieur ».

\subsection*{v. 2469--2471}

\textbf{\textit{antant}}, impératif de \textit{entendre} : qui n'a pas ici le sens de base de « percevoir par l'ouie » mais doit être renforcé « prêter attention, écouter attentivement »

\textbf{\textit{demande}} et \textbf{\textit{quiert}} ont pour actant \textit{tes freres}, soit \textit{Alexandre} (\textit{stricto sensu} sujet pronominal effacé en place 3)

\textbf{\textit{la soe chose}} : encore une fois combinaison d'un caractérisant possessif et d'un article, ici défini $\Rightarrow$ choisir de faire porter l'effort de la traduction sur le possessif est ici logique, car c'est son bien que défend Alexandre par la bouche de son envoyé.

\subsection*{v. 2472--2473}

\textbf{\textit{iert}} : futur P3 \textit{estre}

\textbf{\textit{soe}}, CSSF du caractérisant possessif : ici construit en attribut du sujet \textit{Constantinoble}

\textbf{\textit{tenir}} au sens fort de « diriger ».

\subsection*{v. 2474--2475}

\textbf{\textit{eüst}} : subj. impft P3 \textit{avoir} à valeur de potentiel.

\subsection*{v. 2476--2478}

\textbf{\textit{soi acorder}}, v. pron. : « trouver un accord », « se réconcilier » (DECT)

\textbf{\textit{si}} : adv. de phrase saturant la place 1 auquel on peut trouver ici une nuance consécutive

\textbf{\textit{rant}} : impératif de \textit{rendre}

\textbf{\textit{pes}}, s.f. : « paix »

\textbf{\textit{les}} : IP2 de \textit{laire/layer} « laisser ».

\subsection*{v. 2479}

\textbf{\textit{biax dolz ami}} : apostrophe marquant le changement d'interlocuteur dans le discours direct. À traduire en évitant le calque.

\subsection*{v. 2480--2481}

\textbf{\textit{ies}} : IP2 \textit{estre}

\textbf{\textit{qui\ldots aporté}} : proposition relative dont l'antécédent est le PPC représentant Acorïonde « toi qui\ldots ».

\subsection*{v. 2482--2483}

Noter que l'adverbe négatif \textit{ne} peut saturer la place 1.

\subsection*{v. 2484}

\textbf{\textit{croi}} : IP1 \textit{croire}

\textbf{\textit{as < a+les}} : enclise de l'article défini et de la préposition \textit{a}.

\subsection*{v. 2485--2486}

\textbf{\textit{vis}} : (\textit{vif+s}, variante combinatoire déjà notée), CSSM de \textit{vif} « vivant »

\textbf{\textit{jel < je+le}} : enclise du PPS et du PPC

\textbf{\textit{nel< ne+le}} : enclise de l'adverbe négatif et du PPC

\textbf{\textit{cresrai}} : forme inhabituelle du futur P1 de \textit{croire}.

\subsection*{v. 2487}

\textbf{\textit{piece a}}, loc. adv. : « il y a longtemps », « depuis un certain temps »

\textbf{\textit{poise}} : IP3 \textit{peser} dont le sujet est le pronom démonstratif \textit{ce}.

\subsection*{v. 2488}

\textbf{\textit{diës}} : sub.présent P3 \textit{dire}.

\subsection*{v. 2490--2491}

\textbf{\textit{covient}} : forme impersonnelle

\textbf{\textit{li}} : PPC renvoyant à Alexandre

\textbf{\textit{doigne}} : sub.présent P1 \textit{donner}

\textbf{\textit{assez}}, adv. : au sens de « un grand nombre de ».

\subsection*{v. 2492--2493}

\textbf{\textit{fos}} : (\textit{fol+s} avec vocalisation non marquée dans la graphie), adj. CSSM, « fou »

\textbf{\textit{se / s'}} : conj. de subordination à valeur hypothétique

\textbf{\textit{iert}} : futur P3 \textit{estre}

\textbf{\textit{ne ja}} : négation renforcée à valeur absolue « ne\ldots pas du tout, ne\ldots absolument pas ».

\subsection*{v. 2494--2495}

\textbf{\textit{de}}, préposition : qui a ici sa valeur latine (\textit{de} + abl.) « au sujet de\ldots, quant à »

\textbf{\textit{nus}} : (\textit{nul+s} CSSM) sujet de \textit{iert tenanz}, « sera détenteur »

\textbf{\textit{ne\ldots ja}} : négatif renforcée à valeur temporelle « ne\ldots jamais ».

\subsection*{v. 2496--2497}

\textbf{\textit{cil}} : pronom démonstratif renvoyant à Acorïonde

\textbf{\textit{ot}} : IP3 \textit{oïr}

\textbf{\textit{la responsse l'empereor}} : complément déterminatif du nom construit en CRA

\textbf{\textit{avenant}}, p.présent : « convenable ».

\end{document}
