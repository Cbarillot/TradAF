\documentclass[12pt,a4paper]{article}
\usepackage[utf8]{inputenc}
\usepackage[T1]{fontenc}
\usepackage[french]{babel}
\usepackage{tradaf}

\title{Exemple de Traduction Comparée\\Ancien Français -- Français Contemporain}
\author{Utilisant le package TradAF}
\date{\today}

\begin{document}

\maketitle

\section*{Introduction}

Ce document illustre l'utilisation du package \texttt{tradaf} pour créer des traductions comparées de textes en ancien français vers le français contemporain, avec des notes de traduction catégorisées.

\subsection*{Catégories de notes disponibles}

\begin{itemize}
  \item \textcolor{lexiquecolor}{\textbf{Lexique}} : notes sur le vocabulaire et le sens des mots
  \item \textcolor{syntaxecolor}{\textbf{Syntaxe}} : notes sur la structure des phrases
  \item \textcolor{morphologiecolor}{\textbf{Morphologie}} : notes sur les formes grammaticales
  \item \textcolor{otherscolor}{\textbf{Autres}} : notes diverses (contexte historique, culturel, etc.)
\end{itemize}

\newpage

% Example 1: All categories shown
\begin{translationsection}{Extrait 1 : La Chanson de Roland (avec toutes les catégories)}

\begin{translation}

\oldfrench
\verseline Carles li reis, nostre emperere magnes,\\
\verseline Set anz tuz pleins ad estét en Espaigne :\notelexique{«magnes» signifie «grand», du latin \textit{magnus}}\\
\verseline Tresqu'en la mer cunquist la tere altaigne.\notesyntaxe{Inversion sujet-verbe typique de l'ancien français}\\
\verseline N'i ad castel ki devant lui remaigne\notemorphologie{«remaigne» : subjonctif présent, 3e pers. du verbe «remaindre» (rester)}\\
\verseline Murs ne citét n'i est remés a fraindre,\noteother{Référence à la conquête de l'Espagne par Charlemagne, récit légendaire}\\

\modernfrench
\verseline Charles le roi, notre grand empereur,\\
\verseline Sept ans entiers a été en Espagne :\\
\verseline Jusqu'à la mer il conquit la terre altière.\\
\verseline Il n'y a château qui tienne devant lui,\\
\verseline Mur ni cité qui reste à détruire,\\

\end{translation}

\end{translationsection}

\newpage

% Example 2: Only lexique and syntaxe
\showonlycategories{lexique,syntaxe}

\begin{translationsection}{Extrait 2 : Même texte (uniquement lexique et syntaxe)}

\begin{translation}

\oldfrench
\verseline Carles li reis, nostre emperere magnes,\\
\verseline Set anz tuz pleins ad estét en Espaigne :\notelexique{«magnes» signifie «grand», du latin \textit{magnus}}\\
\verseline Tresqu'en la mer cunquist la tere altaigne.\notesyntaxe{Inversion sujet-verbe typique de l'ancien français}\\
\verseline N'i ad castel ki devant lui remaigne\notemorphologie{«remaigne» : subjonctif présent, 3e pers. du verbe «remaindre» (rester)}\\
\verseline Murs ne citét n'i est remés a fraindre,\noteother{Référence à la conquête de l'Espagne par Charlemagne, récit légendaire}\\

\modernfrench
\verseline Charles le roi, notre grand empereur,\\
\verseline Sept ans entiers a été en Espagne :\\
\verseline Jusqu'à la mer il conquit la terre altière.\\
\verseline Il n'y a château qui tienne devant lui,\\
\verseline Mur ni cité qui reste à détruire,\\

\end{translation}

\end{translationsection}

\newpage

% Example 3: Only morphologie
\showonlycategories{morphologie}

\begin{translationsection}{Extrait 3 : Le Roman de Renart (uniquement morphologie)}

\begin{translation}

\oldfrench
\verseline Seignors, oï avez\notemorphologie{«oï» : participe passé de «oïr» (entendre)} maint conte\\
\verseline Que maint conterre vous raconte\notemorphologie{«conterre» : forme ancienne de «conteur»}\\
\verseline Comment Paris ravi Elaine\notemorphologie{«ravi» : passé simple de «ravir» (enlever)}\\
\verseline Le mal qu'il en ot et la paine\notemorphologie{«ot» : passé simple de «avoir», 3e personne}\\

\modernfrench
\verseline Seigneurs, vous avez entendu maints contes\\
\verseline Que maint conteur vous raconte\\
\verseline Comment Pâris enleva Hélène\\
\verseline Le mal qu'il en eut et la peine\\

\end{translation}

\end{translationsection}

\newpage

% Reset to show all categories
\showallcategories

\begin{translationsection}{Extrait 4 : Prose (toutes catégories)}

\begin{translation}

\oldfrench
En cel tens\notelexique{«cel» : forme ancienne de «ce», du latin \textit{ille}} que li rois Artus\noteother{Le roi Arthur, figure légendaire de la matière de Bretagne} tenoit\notemorphologie{«tenoit» : imparfait de l'indicatif, 3e pers. sing.} la terre de Logres\noteother{«Logres» : nom ancien de l'Angleterre}, fu avenuz\notemorphologie{«avenuz» : participe passé de «avenir» (arriver)} une chose merveilleuse\notelexique{«merveilleuse» : digne d'émerveillement, prodigieuse}.

\modernfrench
En ce temps où le roi Arthur tenait la terre de Logres, il arriva une chose merveilleuse.

\end{translation}

\end{translationsection}

\section*{Instructions d'utilisation}

\subsection*{Filtrage des catégories}

\begin{itemize}
  \item \texttt{\textbackslash showallcategories} : afficher toutes les catégories
  \item \texttt{\textbackslash showonlycategories\{lexique,syntaxe\}} : afficher uniquement certaines catégories
  \item \texttt{\textbackslash showlexique}, \texttt{\textbackslash hidelexique} : afficher/masquer la catégorie lexique
  \item Commandes similaires pour \texttt{syntaxe}, \texttt{morphologie}, et \texttt{others}
\end{itemize}

\subsection*{Commandes principales}

\begin{itemize}
  \item \texttt{\textbackslash begin\{translation\}} : environnement pour le texte parallèle
  \item \texttt{\textbackslash oldfrench} : basculer vers la colonne ancien français
  \item \texttt{\textbackslash modernfrench} : basculer vers la colonne français moderne
  \item \texttt{\textbackslash notelexique\{...\}} : ajouter une note de lexique
  \item \texttt{\textbackslash notesyntaxe\{...\}} : ajouter une note de syntaxe
  \item \texttt{\textbackslash notemorphologie\{...\}} : ajouter une note de morphologie
  \item \texttt{\textbackslash noteother\{...\}} : ajouter une autre note
  \item \texttt{\textbackslash verseline} : numéroter les vers (optionnel)
\end{itemize}

\end{document}
